% !TeX root = ../main.tex

\chapter{绪论}
\label{cha:intro}

一直以来,很多物理学家在尝试找到一个可以统一自然界和整个宇宙的所有物质和现象的“终极理论”,
并希望它能解释所有已知的物质和相互作用及其基本结构,小到质子中子,大到银河系或者整个宇宙;
它能追溯宇宙的起源,关于宇宙的存在和宇宙是怎么产生的;
它也能对某些现象进行预测等等。
早在十九世纪初期,随着爱因斯坦狭义相对论的提出~\cite{SRE}和量子力学~\cite{QMP}的发现,经典物理的基础被动摇。
狭义相对论迫使科学家改变了他们对时间和空间的认知,时间和空间是不可分割的交织在一起的,
是单一现实的不同两个方面,即时空;量子力学让科学家们看到微观世界的运动规律和宏观世界如此不一样,
粒子的行为表现的像场,而场的行为也类似于粒子。
到十九世纪中期,狄拉克将量子力学和狭义相对论完美的融合在了一起~\cite{QMP},
随后科学家们将它们与经典场论结合起来形成了一种功能强大的物理框架,量子场论~\cite{QFT1}。
同时,以量子场论作为基本框架,标准模型~\cite{SM2}也随之萌芽,一个与“终极理论”最为接近的物理模型,
十九世纪后半叶便是标准模型随着各式的高能物理实验一起发展、逐渐完善的过程。
而对称性在标准模型的发展过程中一直扮演着重要角色,
这起源于早期外尔发现了电磁相互作用中的规范对称性~\cite{WEYL},在科学家们发现夸克之后,
一种隐藏在夸克内部的“颜色”对称性逐渐浮出水面~\cite{SM0},
对应于一种束缚质子和中子中夸克的作用力,强相互作用。
随后的一个重大突破便是科学家们意识到对称性或许可以自发的破坏~\cite{SM4,SM5,SM6},
到后来二十世纪末期W、Z玻色子和二十一世纪初期“上帝粒子”希格斯玻色子的发现~\cite{ZBOSON,WBOSON1,WBOSON2,ATLASHIGGS,CMSHIGGS},
验证了对称性自发破缺这个设想的正确性,这也代表着标准模型的巨大成功,
到目前为止,所有已知的粒子都具有该模型所精确预测的性质。
与此同时,科学家们对标准模型也提出了很多新问题,
为什么不包含引力?暗物质从何而来?为什么夸克分为六个味而且质量差异如此之大?等等,
因此对超出标准模型的新物理的研究仍然具有重要的意义。

目前,欧洲核子研究中心(The European Organization for Nuclear Research, CERN)的大型强子对撞机(The Large Hadron Collider, LHC)~\cite{Evans:2008zzb}作为全世界最大的粒子物理实验室,开展着一系列该领域的前沿研究,
其中两个重要的方面就是与希格斯玻色子相关的性质测量和超出标准模型的新物理的寻找,
这篇论文的研究工作便是在这个背景下完成的,
论文工作主要集中在两个方面:
基于LHC的ATLAS(A Toroidal LHC ApparatuS)探测器~\cite{PERF-2007-01}中高横动量希格斯玻色子衰变到双b夸克的标定和在双喷注末态的不变质量谱中寻找超出标准模型的新物理。
具体的章节安排为:
第~\ref{cha:intro}~章简要的介绍研究背景;
第~\ref{cha:Theory}~章介绍研究工作的理论背景,包括标准模型、希格斯玻色子的性质、
研究工作的动机和与研究工作相关的超出标准模型的新物理模型;
第~\ref{cha:EXP}~章着重于实验技术,首先会简要介绍一下LHC,然后详细介绍LHC上的ATLAS探测器及其各个组成部分,
随后简要介绍模拟实现技术和ATLAS探测器中物理对象的重建技术;
第~\ref{cha:ML}~章会简要介绍高能物理实验领域常用的两种机器学习算法:神经网络和提升决策树,其中第~\ref{cha:Xbb}~章的研究工作便是以神经网络为基础的;
第~\ref{cha:Xbb}~章详细介绍了高横动量希格斯玻色子衰变到双b夸克的标定算法的开发动机、研究现状、设计思路、性能及其与ATLAS合作组已有算法的对比等,%及其与ATLAS合作组已有算法的对比等,
这部分研究工作已经以公告文章(ATLAS public note)的形式发表出来~\cite{ATL-PHYS-PUB-2020-019};
第~\ref{cha:Dijet}~章详细介绍了物理分析在双喷注末态的不变质量谱中寻找超出标准模型的新物理的动机、研究现状、技术手段和实验结果等,
这部分研究工作的结果最近发表在高能物理杂志(Journal of High Energy Physics, JHEP)上面~\cite{DijetPaper};%并且我在其中作为分析的主力成员之一
最后第~\ref{cha:Summary}~章是研究工作的总结和展望部分。












