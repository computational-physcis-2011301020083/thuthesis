% !TeX root = ../main.tex

\begin{denotation}[3cm]
\item[CERN] 欧洲核子研究中心(The European Organization for Nuclear Research)
\item[LHC] 大型强子对撞机(The Large Hadron Collider)
\item[ATLAS] 环形LHC设备(A Toroidal LHC ApparatuS)
\item[CMS] 紧凑型$\mu$子螺线管(The Compact Muon Solenoid)
\item[SM] 标准模型(The Standard Model)
\item[QFT] 量子场论(Quantum Field Theory)
\item[q] 夸克(Quark)
\item[g] 胶子(Gluon)
\item[CPT] 电荷、宇称和时间反演对称(Charge、parity and time reversal symmetry)
\item [QED] 量子电动力学(Quantum electrodynamics)
\item[QCD] 量子色动力学(Quantum chromodynamics)
\item[EW] 电弱理论(Electroweak theory)
\item[CKM] Cabibbo–Kobayashi–Maskawa
\item[SSB] 对称性自发破缺(Spontaneous symmetry breaking)
\item[BR] 分支比(Branching ratio)
\item[BSM] 超出标准模型的新物理(Physics beyond the Standard Model )
\item[DM] 暗物质(Dark Matter)
\item[SSM] 连续标准模型(The Sequential Standard Model)
\item[KK] Kaluza-Klein
\item[G] 引力子(Graviton)
\item[ggF] 胶子融合(Gluon–gluon fusion)
\item[VBF] 矢量玻色子融合(Vector-boson fusion)
\item[jet] 喷注
\item[b-jet] 含b强子的喷注
\item[LR-jet] 距离参数R=1的大喷注(Large-R jet)
\item[subjet] 包含在距离参数R=1的大喷注内部的喷注
\item[MC] Monte Carlo
\item[PDF] 部分子分布函数(Parton Distribution Functions)
\item[PD] 像素探测器(The Pixel Detector)
\item[SCT] 半导体径迹探测器(The SemiConductor Tracker)
\item[DNN] 深度学习神经网络(Deep Neural Network)
\item[BDT] 提升决策树(Boosted Decision Tree)
\item[JSS] LR-jet的结构变量(Jet substructure)
\item[FR-jet] 由距离参数R=0.2的anti-$k_t$算法基于径迹重建出来的喷注(Fixed radius track jet)
\item[VR-jet] 由距离参数R可变的anti-$k_t$算法基于径迹重建出来喷注(Variable radius track jet)
\item[TH-jet] 由anti-$k_t$算法基于产生子层面真实粒子的径迹重建出来的喷注(Truth jet)
\item[LO] 领头阶(Leading order)
\item[JES] 喷注能量刻度(Jet energy scale)
\item[JER] 喷注分辨率(Jet energy resolution)
\item[dijet] 双喷注
\end{denotation}



% % 也可以使用 nomencl 宏包:

% \printnomenclature[3cm]

% \nomenclature{HPC}{高性能计算 (High Performance Computing)}
% \nomenclature{cluster}{集群}
% \nomenclature{Itanium}{安腾}
% \nomenclature{SMP}{对称多处理}
% \nomenclature{API}{应用程序编程接口}
% \nomenclature{PI}{聚酰亚胺}
% \nomenclature{MPI}{聚酰亚胺模型化合物,N-苯基邻苯酰亚胺}
% \nomenclature{PBI}{聚苯并咪唑}
% \nomenclature{MPBI}{聚苯并咪唑模型化合物,N-苯基苯并咪唑}
% \nomenclature{PY}{聚吡咙}
% \nomenclature{PMDA-BDA}{均苯四酸二酐与联苯四胺合成的聚吡咙薄膜}
% \nomenclature{$\increment G$}{活化自由能 (Activation Free Energy)}
% \nomenclature{$\chi$}{传输系数 (Transmission Coefficient)}
% \nomenclature{$E$}{能量}
% \nomenclature{$m$}{质量}
% \nomenclature{$c$}{光速}
% \nomenclature{$P$}{概率}
% \nomenclature{$T$}{时间}
% \nomenclature{$v$}{速度}
