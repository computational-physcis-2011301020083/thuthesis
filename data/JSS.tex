第一组变量是LR-jet的能量关联比,
为了表征LR-jet内部的辐射结构,
利用LR-jet组成成分的能量和相对角距,
研究~\cite{JSS1,JSS2}引入了一般性的能量关联函数:
\begin{equation} 
\label{eq:JSS1}
 \begin{split}
E_{CF1}&=\sum_{i\in J} p_{T}^i
\\
E_{CF2}(\beta)&=\sum_{i<j\in J} p_T^i p_T^j (R_{ij})^\beta
 \\
E_{CF3}(\beta)&=\sum_{i<j<k\in J} p_T^i p_T^j (R_{ij})^\beta
 \end{split}
\end{equation}
其中i、j、k是LR-jet的组份,$p_{T}^i$是第i个组份的横动量,
$R_{ij}$是第i个组份与第j个组份之间的角距,
$\beta$取正整数,用于给组份的角距加权。
接着可以定义两组比率:
\begin{equation} 
\label{eq:JSS2}
 \begin{split}
e_2^{(\beta)}&=\frac{E_{CF2}(\beta)}{E_{CF1}^2}
\\
e_3^{(\beta)}&=\frac{E_{CF3}(\beta)}{E_{CF1}^3}
 \end{split}
\end{equation}
然后可以更近一步的定义:
\begin{equation} 
\label{eq:JSS3}
 \begin{split}
C_2^{(\beta)}&=\frac{e_3^{(\beta)}}{(e_2^{(\beta)})^2}
\\
D_2^{(\beta)}&=\frac{e_3^{(\beta)}}{(e_2^{(\beta)})^3}
 \end{split}
\end{equation}
研究表明它们对于LR-jet内部结构的识别有一定
作用,
%用处,
其中$e_3=e_3^{(\beta=1)}$、$C_2=C_2^{(\beta=1)}$和$D_2=D_2^{(\beta=1)}$
在W玻色子标定和t夸克标定中都表现出一定的区分能力~\cite{JSCD2}。

第二组变量用于表征LR-jet中的subjet多样性~\cite{JSS3,JSS4},
基于$k_t$算法~\cite{JSS14,JSTAU},可以将LR-jet的组份会聚成N个subjet,
然后可以定义一个变量$\tau_N$,用于量化这N个subjet能够多么合适的描述LR-jet:
\begin{equation} 
\label{eq:JSS4}
\tau_N = \frac{ \sum_{i\in J} p_T^i min \{ \Delta R_{1i},R_{2i},\dots ,R_{Ni} \} }{R\sum_{i\in J} p_T^i}
\end{equation}
其中$p_T^i$是第i个组份的横动量,
$min \{ \Delta R_{1i},R_{2i},\dots ,R_{Ni}, \}$是第i个组份与距离这个组份最近的subjet轴线的角距,
R=1是LR-jet的距离参数。
当$\tau_N \approx 0$时,表明LR-jet中的每个组份都有与之共线的subjet相对应,
%而
当$\tau_N \approx 1$时,表明在
%除了
靠近subjet轴线外的其他方向仍有显著的能量辐射,
也就是这N个subjet不能合适的描述这个LR-jet。
为了区分两体衰变和三体衰变,
引入比率$\tau_{21}=\tau_2/\tau_1$和$\tau_{32}=\tau_3/\tau_2$,
它们在W玻色子标定和t夸克标定中也表现出一定的区分能力~\cite{JSS3}。

第三组变量KtDR也是基于对LR-jet组份
%进行
的
重新会聚,
在距离参数R=0.4的情况下可以通过$k_t$算法将LR-jet的组份会聚成两个subjet,
而变量KtDR便是这两个subjet轴线之间的角距~\cite{JSS14}。
对于%LR-jet
质量为m横动量为$p_T$的LR-jet的两体衰变,
这个变量有一个特征值$KtDR\approx 2m/p_T$。

第四组变量Qw同样是基于对LR-jet组份
%进行
的重新会聚,
%也可以
通过$k_t$算法将LR-jet的组份会聚成三个subjet,
变量$Q_w$表示其中两个subjet的不变质量最小的值~\cite{JSS8}。
它是针对三体衰变过程优化的变量,
在由t夸克衰变而来的W玻色子的标定中有一定的作用。

第五组变量是基于$k_t$算法在会聚LR-jet组份过程中所形成的中间
%过程
subjet,
定义两个与式~\ref{eq:TOPOSN}~类似的分离比例$\sqrt{d_{12}}$和$\sqrt{d_{23}}$~\cite{JSS10}:
\begin{equation} 
\label{eq:JSS45}
\sqrt{d_{ij}}=min(p_T^i,p_T^j)\Delta R_{ij}
\end{equation}
$\sqrt{d_{12}}$代表$k_t$算法在最后一步合并过程中从2个subjet到1个subjet的分离比例,
$\sqrt{d_{23}}$代表$k_t$算法在倒数第二步合并过程中从3个subjet到2个subjet的的分离比例。
其中$p_T^i$是利用$k_t$算法会聚过程中的第i个subjet的横动量,
是$\Delta R_{ij}$第i个subjet和第j个subjet的角距。
对于高横动量强子型衰变的t夸克,
$\sqrt{d_{12}}$的期望值约为t夸克质量的一半$m_t/2$,
对于高横动量强子型衰变的W玻色子,
$\sqrt{d_{23}}$的期望值约为W玻色子质量的一半$m_W/2$。
但是对于来自QCD过程的轻夸克或者胶子辐射,
分离比例$\sqrt{d_{12}}$和$\sqrt{d_{23}}$都较小~\cite{JSD12}。
在此基础上可以定义一个与分离比例相关的变量~\cite{JSS8}:
\begin{equation} 
\label{eq:JSS46}
Z_{cut12}=\frac{d_{12}}{d_{12}+m^2}
\end{equation}
其中$d_{12}$是式~\ref{eq:JSS45}~中定义的最后一步合并过程中的分离比例,
m是最后一步合并过程中由第1个和第2个subjet重新会聚而成的subjet的质量。
对于两体衰变,m代表LR-jet的质量,
$Z_{cut12}$也是和$d_{12}$类似的子结构变量,
但是通过对m进行归一化,可以减少它对LR-jet质量的依赖性。


第六组变量是二阶与零阶Fox-Wolfram矩的比率~\cite{JSS5},
最开始提出的Fox-Wolfram矩是用来表征正负电子碰撞事例的形状,
随后也用来研究LR-jet的形状,
它是在LR-jet的质心系下定义的,
首先将LR-jet的组份利用洛伦兹变换提升到质心系:
\begin{equation} 
\label{eq:JSS5}
 \begin{split}
 \tilde{E}&=\gamma(E+\vec{\beta}\cdot \vec{p})
\\
\tilde{\vec{p}}&=\vec{p}+\frac{\gamma-1}{|\vec{\beta}|^2} (\vec{\beta}\cdot \vec{p}) \vec{\beta}+\gamma E\vec{\beta}
 \end{split}
\end{equation}
其中$\vec{\beta}=\vec{p}_J/E_J$是LR-jet的速度矢量与真空光速的比值,
$\gamma$是洛伦兹变换因子。
接下来可以在质心系下定义l阶Fox-Wolfram矩:
\begin{equation} 
\label{eq:JSS6}
H_l=\sum_{1\le i<j\le n_J} \frac{|\tilde{\vec{p_i}}| |\tilde{\vec{p_j}}|}{s} P_l (\cos \theta_{ij})
\end{equation}
其中$\tilde{\vec{p_i}}$是LR-jet第i个组份的动量,
$s=(\sum_i E_i)^2$是组份的能量和的平方,
$P_l$是第l阶勒让德多项式,
$\theta_{ij}$是第i个组份与第j个组份之间的夹角,
上述与LR-jet组份相关的变量都是质心系下的变量。
随后可以定义归一化的二阶Fox-Wolfram矩,即二阶与零阶Fox-Wolfram矩的比率:
\begin{equation} 
\label{eq:JSS61}
R_2^{FW}=\frac{H_2}{H_0}=\frac{\sum_{1\le i<j\le n_J} |\tilde{\vec{p_i}}| |\tilde{\vec{p_j}}|  (3\cos^2 \theta_{ij}-1 )/2 }
{\sum_{1\le i<j\le n_J} |\tilde{\vec{p_i}}| |\tilde{\vec{p_j}}|}
\end{equation}
归一化的二阶Fox-Wolfram矩对在质心系下对称的结构即背对背的辐射结构比较敏感,
使得它可以用于区分两体衰变和三体衰变。

第七组变量平面度A也是一个形状变量,
它也是在LR-jet的质心系下定义的~\cite{JSS6},
将LR-jet的组份利用洛伦兹变换~\ref{eq:JSS5}~提升到质心系之后,
首先定义一个$3\times 3$的球形矩阵:
\begin{equation} 
\label{eq:JSS62}
S^{k,l}=\frac{ \sum_{1\le i<j\le n_J} \tilde{p}_i^k  \tilde{p}_j^l  }
{\sum_{i=1}^{n_J} |\tilde{\vec{p}}_i|^2}
,\quad k,l \in \{x,y,z\}
\end{equation}
其中$n_J$是LR-jet的组份数量,$\tilde{\vec{p}}_i$是第i个组份的动量,
$\tilde{p}_j^l$和$\tilde{p}_i^k$是第i个组份的动量分量,
上述与LR-jet组份相关的变量都是质心系下的变量。
这个球形矩阵$S^{k,l}$有三个本征值$\lambda_1\le \lambda_2\le \lambda_3$,它们的和为1,
利用本征值$\lambda_3$可以定义平面度:
\begin{equation} 
\label{eq:JSS63}
A=\frac{3\lambda_3}{2}
\end{equation}
它满足$0\le A \le 1/2$,
在质心系下,对于QCD过程中轻夸克或胶子各向同性的弥散辐射,$A\approx 1/2$,
对于背对背的两体衰变,%会使得
$A \approx 0$。


第八组变量倾斜度$a_3$由以下式子定义~\cite{JSS13,JSA3}:
\begin{equation} 
\label{eq:JSS7}
a_3=\frac{1}{m_J} \sum_{i=1}^{n_J} E_i \sin^{-2} \theta_i (1-\cos \theta_i)^3
\end{equation}
其中$m_J$为LR-jet的质量,$n_J$是其组份数量,
$E_i$是第i个组份的能量,$\theta_i$是第i个组份相对于LR-jet轴线的夹角。
$a_3$测量了广角辐射在LR-jet中所占的比重。
对于对称的两体衰变,$a_3$趋向于较小的值,
对于
QCD过程中
%非共振态的
轻夸克或者胶子辐射,$a_3$趋向于较大的值。

%\mathbf
第九组变量平面流$\mathcal{P}$用于衡量LR-jet中能量
在垂直于其轴线方向的平面上的分布,
以及沿平行于其轴线方向的分布情况~\cite{JSS11}。
%沿垂直于其轴线方向的平面和平行于其轴线方向的分布情况~\cite{JSS11}。
将LR-jet动量$\vec{P}$以及其组份的动量$\vec{p}_i$先绕z轴旋转$-\phi$:
\begin{equation} 
\label{eq:JSS8}
\mathbf{R}_{\phi}=
\begin{bmatrix}
\cos \phi & \sin\phi & 0\\
-\sin\phi & \cos\phi & 0\\
 0&0&1
\end{bmatrix}
\end{equation}
然后再绕y轴旋转$-\theta$:
\begin{equation} 
\label{eq:JSS9}
\mathbf{R}_{\eta}=
\begin{bmatrix}
\cos \theta & 0 & -\sin\theta\\
0 & 1 & 0\\
 \sin\theta&0& \cos\theta
\end{bmatrix}
=
\begin{bmatrix}
\tanh\eta & 0 & -\sech\eta\\
0 & 1 & 0\\
 \sech\eta&0& \tanh\eta
\end{bmatrix}
\end{equation}
其中$\phi$和$\eta$分别是LR-jet轴线的方位角和赝快度。
两次旋转操作之后,%使得
LR-jet的动量方向会指向z轴正向:
\begin{equation} 
\label{eq:JSS10}
\tilde{\vec{P}}=\mathbf{R}_{\eta}\mathbf{R}_{\phi}\vec{P}=
\begin{bmatrix}
\tanh\eta\cos \phi & \tanh\eta\sin\phi & -\sech\eta\\
-\sin\phi & \cos\phi & 0\\
  \sech\eta\cos\phi &\sech\eta\sin\phi & \tanh\eta
\end{bmatrix}
\begin{bmatrix}
P_x \\
P_y \\
P_z
\end{bmatrix}
=
\begin{bmatrix}
0 \\
0 \\
|\vec{P}|
\end{bmatrix}
\end{equation}
这样便于研究LR-jet组份的动量$\vec{p}_i$
%沿着
在垂直于LR-jet轴线方面的平面上的分布情况。
利用上述旋转矩阵将其组份i随着LR-jet轴线变换:
%之后:
\begin{equation} 
\label{eq:JSS11}
\tilde{\vec{p}}_i=\mathbf{R}_{\eta}\mathbf{R}_{\phi}\vec{p}_i
\end{equation}
其中$\vec{p}_i$是LR-jet第i个组份的动量,
接着可以定义一个$2\times 2$的动量矩阵$I_E$:
\begin{equation} 
\label{eq:JSS12}
I_E^{kl}=\frac{1}{m_J} \sum_{i=1}^{n_J} \frac{\tilde{p}_i^k\tilde{p}_i^l}{E_i}, \quad k,l \in \{x,y\}
\end{equation}
其中$m_J$是LR-jet的质量,$n_J$是其组份数量,
$E_i$是第i个组份的能量,$\tilde{p}_i^k$是第i个组份在旋转之后的动量分量。
随后平面流可以通过矩阵$I_E$的两个本征值$\lambda_1$和$\lambda_2$来定义:
\begin{equation} 
\label{eq:JSS13}
\mathcal{P}=\frac{3\det{I_E}}{tr(I_E)^2}=\frac{4\lambda_1\lambda_2}{(\lambda_1+\lambda_2)^2}
\end{equation}
如果能量在垂直于LR-jet轴线方向的平面上分布比较均匀,
会使$\lambda_1\approx\lambda_2\approx 1/2$,从而平面流$\mathcal{P}\approx 1$;
如果能量沿平行于LR-jet轴线的方向分布比较均匀,
则$\lambda_1\gg \lambda_2$,平面流$\mathcal{P}\approx 0$。
由于两体衰变趋向于产生沿着LR-jet轴线方面的线性能量分布,
而
%非共振态的
QCD过程中轻夸克或者胶子
%辐射
更趋向于产生
各向同性的弥散辐射,
%弥散的各向同性的能量辐射,
因此平面流$\mathcal{P}$
可以作为一个很好的区分变量。
%非常适合区分这两者。










































