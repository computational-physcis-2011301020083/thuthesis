对于$m_{jj}$的分bin,我们希望bin的
宽度越小越好,
%总数越大越好,
%达到最大,
从而能够探测到尽可能窄的新共振态,
但是,为了限制bin到bin之间事例的迁移带来的影响,bin的宽度需要大于探测器的分辨率。
%因此首先我们需要找到探测器的分辨率,然后决定bin宽。

为了找到探测器的分辨率,
我们在多个$m_{jj}^{truth}$区间,对
%比值
$m_{jj}^{reco}/m_{jj}^{truth}$的分布用一个高斯分布进行拟合,
然后,探测器的分辨率R可以由拟合出来的标准差除以均值$R=\sigma/\mu$得到。
其中$m_{jj}^{reco}$是
从来自QCD过程的MC样本中
%由MC模拟
重建出来的事例中两个领头jet的不变质量,
利用与
%MC模拟重建出来的两个领头jet
其相关联的TH-jet,
可以计算相应的不变质量$m_{jj}^{truth}$,
其中TH-jet是通过产生子层面的真实粒子构建的,
与第~\ref{sec:XbbORFT}~小节中提高的类似,
%称为真实的jet,
%为了与探测器响应相对应,要求这些真实粒子的寿命大于10ps,且不包含$\mu$和中微子,
%随后以这些真实粒子为单元,
基于产生子层面真实粒子的径迹,
利用第~\ref{sec:JET}~小节中与SR-jet相同的重建算法构建TH-jet,
随后通过
%第~\ref{sec:JET}~小节描述的
“幻影”关联
%~\cite{Cacciari:2008gn}
%的过程
将
MC样本中的jet
%MC模拟重建出来的jet
与TH-jet的关联起来。

接下来,将探测器的分辨率R随着$m_{jj}^{truth}$变化的分布用函数:
\begin{equation} 
\label{eq:MJJR1}
R(m_{jj}^{truth})=\frac{a}{m_{jj}^{truth}}+\frac{b}{\sqrt{m_{jj}^{truth}}}+c
\end{equation}
进行拟合,用于对分辨率R随$m_{jj}^{truth}$的变化建模。

在领头jet横动量$p_T>150GeV$的初始筛选条件下,
边界946GeV略高于MC样本中$m_{jj}$分布的峰值,而且远低于分析中$m_{jj}$的截断值。
于是,我们设定一个bin初始边界$m_{initial}=946GeV$之后,
用以下方法决定其他更高质量区的bin边界:
\begin{enumerate}
  \item 对于给定的bin初始边界$m_{initial}$,计算相应的分辨率$R(m_{initial})$,
  然后推断该bin中心值为$m_{centre}=m_{initial}(1+\frac{1}{2}R(m_{initial}))$;
  \item 由推断的bin中心值$m_{center}$反过来计算bin左边的边界值
  $m_{lower}=m_{centre}(1-\frac{1}{2}R(m_{centre}))$;
  \item 检查$m_{initial}$和$m_{lower}$之间的相对差异是否小于0.1\%,
  如果差异大于0.1\%且$m_{lower}>m_{initial}$,则使$m_{centre}=m_{centre}-0.01GeV$,然后重复步骤2中的过程,
 如果差异大于0.1\%且$m_{lower}<m_{initial}$,则使$m_{centre}=m_{centre}+0.01GeV$,然后重复步骤2中过程;
  \item 一旦$m_{initial}$和$m_{lower}$之间的相对差异小于0.1\%,固定$m_{centre}$,并计算该bin右边的边界值
  $m_{upper}=m_{centre}(1+\frac{1}{2}R(m_{centre}))$,并四舍五入到最接近的整数GeV;
  \item 然后使$m_{initial}=m_{upper}$,并重复以上过程。
\end{enumerate}












