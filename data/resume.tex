% !TeX root = ../thuthesis-example.tex

\begin{resume}

  \section*{个人简历}

1993 年 03 月 02 日出生于湖北省荆州市公安县。

  2011 年 09 月考入武汉大学 物理科学与技术学院物理学专业,2016 年 07 月本科毕业并获得理学学士学位。

  2016 年 09 月免试进入清华大学理学院物理系攻读理学博士学位至今。


  \section*{在学期间完成的相关学术成果}

  \subsection*{学术论文}

  \begin{achievements}
    \item ATLAS Collaboration. Search for new resonances in mass distributions of jet pairs using $139fb^{-1}$ of pp collisions at $\sqrt{s} = 13 TeV$ with the ATLAS detector [J/OL]. Journal of High Energy Physics, 2020, 2020(3):145-145. (SCI收录, 检索号: WOS:000524489600001)
    \item ATLAS Collaboration. Identification of boosted Higgs bosons decaying into $b\bar{b}$ with Neural Networks and variable radius subjets in ATLAS: ATL-PHYS-PUB-2020-019 [R/OL]. Geneva: CERN, 2020. http://cds.cern.ch/record/2724739.
  
\end{achievements}


%  \subsection*{专利}

%  \begin{achievements}
%    \item 任天令, 杨轶, 朱一平, 等. 硅基铁电微声学传感器畴极化区域控制和电极连接的方法: 中国, CN1602118A[P]. 2005-03-30.
%    \item Ren T L, Yang Y, Zhu Y P, et al. Piezoelectric micro acoustic sensor based on ferroelectric materials: USA, No.11/215, 102[P]. (美国发明专利申请号.)
%  \end{achievements}

\end{resume}
